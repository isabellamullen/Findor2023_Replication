\documentclass{article}
\usepackage{graphicx} % Required for inserting images
\usepackage[margin=1in]{geometry}  
\usepackage{setspace} 
\onehalfspacing

\title{Findor et al. 2023 Replication}
\author{Asad Tariq and Isabella Mullen}
\date{\today}

\begin{document}

\maketitle

\section{Introduction}
\begin{itemize}
\item key information about the authors’ analysis
\end{itemize}

\section{Original Paper}
\subsection{Data: Study 3}
\begin{itemize}
 \item What is the unit of analysis/observation in the model you replicate?  How many are there?  How are they sampled (random sample?  Universe of cases?  Data availability? Something else?)?  How willing are you to assume that observations are conditionally independent? I.e., what are the cases in the dataset?
\item Do you have any concerns about observational independence?
\item What is the dependent variable for the model you are replicating? What is the dependent or outcome variable? Describe its important qualities (binary?  categorical?  truncated in some way? Rare? Bimodal? etc.)
\item What model did the author use to explain this outcome variable?  
\item How many observations are there?
\item How were the data sampled (an internet survey?  All available cases? something else?)
\item Describe the DV.  Include a plot of the distribution of the DV and highlight any important qualities (binary?  categorical?  truncated in some way? Rare? Bimodal? etc.)
\item How did the author handle missing data?
\end{itemize}

\subsection{Model}
\begin{itemize}
\item What do the original authors hope to achieve with the statistical model(s) you focus on? (description?  prediction? causal inference?)
\item What kind of model are you replicating (logit?  probit?  negative binomial? etc.)
\end{itemize}

\subsection{Replication}
\begin{itemize}
    \item include a regression table in your paper in which the coefficient estimates, standard errors, and the number of observations are exactly the same as in the original paper.
\end{itemize}

\section{Additional Model}
\begin{itemize}
\item Describe what additional model or models you want to consider and why.
\item State the additional model(s) you are considering.  The additional model(s) must be fit to the same dependent variable.  Explain why you chose the model(s) you did.
\item Provide a regression table that reports the results from this new model(s).
\end{itemize}

\subsection{Interpretation/ Results}
\begin{itemize}
\item Evaluate which of the models you consider performs better on an out-of-sample predictive basis.  I.e., you will use cross validation to compare your models. 
\item Compare the original model and your new model(s) based on their in- and out-of-sample predictive performance. You will use cross-validation for the latter.
\item Interpret the relationship between an independent variable and the dependent variable in the best-fitting model you found.  Be sure to specify your quantity of interest.
\item Explain what you learned about the data and models you explored 
\item Decide which model you think is best and justify your decision.
\item For the best model, you will describes how a particular independent variable of your choosing relates to the dependent variable.  A good paper will include a carefully described quantity of interest and present the interpretive estimate in table or graphical form that accurately incorporates our uncertainty around the estimated quantity of interest. 
\end{itemize}

\section{Conclusion}
\begin{itemize}
    \item How confident are you in the authors’ conclusions after this exercise? 
    \item What more would you like to see done with this paper?
\end{itemize}

\end{document}
